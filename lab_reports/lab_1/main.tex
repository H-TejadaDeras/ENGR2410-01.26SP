%%%%%%%%%%%%%%%%%%%%%%%%%%%%%%%%%%%%%%%%%%%%%%%%%%%%%%%%%%%
% --------------------------------------------------------
% Tau
% LaTeX Template
% Version 2.4.4 (28/02/2025)
%
% Author: 
% Guillermo Jimenez (memo.notess1@gmail.com)
% 
% License:
% Creative Commons CC BY 4.0
% --------------------------------------------------------
%%%%%%%%%%%%%%%%%%%%%%%%%%%%%%%%%%%%%%%%%%%%%%%%%%%%%%%%%%%

\documentclass[9pt,a4paper,column,twoside]{tau-class/tau}
\usepackage[english]{babel}

%% Spanish babel recomendation
% \usepackage[spanish,es-nodecimaldot,es-noindentfirst]{babel} 

%% Draft watermark
% \usepackage{draftwatermark}

%----------------------------------------------------------
% TITLE
%----------------------------------------------------------

\journalname{Intro Microelectronic Circuits with Laboratory}
\title{Resistors and Resistive Networks}

%----------------------------------------------------------
% AUTHORS, AFFILIATIONS AND PROFESSOR
%----------------------------------------------------------

\author[1]{Elias Tuthill}
\author[1]{Anika Mahesh}
\author[1]{Henry Tejada Deras}

%----------------------------------------------------------

\affil[1]{Franklin W. Olin College of Engineering}

%----------------------------------------------------------
% FOOTER INFORMATION
%----------------------------------------------------------

\institution{Franklin W. Olin College of Engineering}
\footinfo{Intro Microelectronic Circuits with Laboratory}
\theday{February 9, 2026}
\leadauthor{Tuthill et. al.}

%----------------------------------------------------------

\begin{document}
		
    \maketitle 
    \thispagestyle{firststyle} 
    % \tauabstract 
    % \tableofcontents
    % \linenumbers 
    
%----------------------------------------------------------

\section{Experiment 1: Resistance Measurement}
	\subsection{Results}
	\subsection{Analysis}
	\subsection{Discussion}
    The nominal resistance value of the resistor was 1.5k$\Omega$ with a 5\% tolerance. The resistance measured by the SourceMeter was 1.48377 k$\Omega$, which has a percent error of 1.08\% which is less than the 5\% tolerance. However the SourceMeter is still subject to a small error rate because while it aims to approximate an ideal voltmeter, it will not truly be infinite resistance. Additionally the resistor itself is subject to a small error rate because of the manufacturing process. In the curve fit plot, the value is likely more accurate because it is based on multiple values of voltage and current. However, there is still some error because of inaccuracy in measurements and the small resistance in wires. However the fact that the value by the SourceMeter is similar to the value from the curve fit plot suggests that both these methods result in a value that is close to the true resistance of the resistor

	Color Code Resistance Value: 1.5 k$\Omega$ with a 5\% tolerance. Keithley Measured Resistance Value: 1.48377 k$\Omega$

\section{Experiment 2: Resistive Voltage Division}
	\subsection{Results}
	\subsection{Analysis}
	\subsection{Discussion}
	Within a single chip, the standard deviation of the resistance between the isolated resistors was about 0.56\% with a range of 0.73. This is within the $\pm$2\% tolerance of the chip.

\section{Experiment 3: Resistive Current Division}
	\subsection{Results}
	\subsection{Analysis}
	\subsection{Discussion}

\section{Experiment 4: R-2R Ladder Network}
	\subsection{Results}
	\subsection{Analysis}
	\subsection{Discussion}
    \begin{table}
        \centering
        \begin{tabular}{ccc}
            1 & 9.9724 k$\Omega$\\
            2 & 9.9661 k$\Omega$\\
            3 & 9.9465 k$\Omega$\\
            4 & 9.9510 k$\Omega$\\
            5 & 9.9445 k$\Omega$\\
            6 & 9.9475 k$\Omega$\\
            7 & 9.9566 k$\Omega$\\
            8 & 9.9571 k$\Omega$\\
            9 & 9.9556 k$\Omega$\\
            10 & 9.9499 k$\Omega$\\
            11 & 9.9645 k$\Omega$\\
            12 & 9.9632 k$\Omega$\\
            13 & 9.9512 k$\Omega$\\
            14 & 9.9515 k$\Omega$\\
            15 & 9.9562 k$\Omega$\\
            16 & 9.9518 k$\Omega$\\
        \end{tabular}
        \caption{Caption}
        \label{tab:placeholder}
    \end{table}

%----------------------------------------------------------

\end{document}