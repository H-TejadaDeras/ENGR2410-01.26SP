%%%%%%%%%%%%%%%%%%%%%%%%%%%%%%%%%%%%%%%%%%%%%%%%%%%%%%%%%%%
% --------------------------------------------------------
% Tau
% LaTeX Template
% Version 2.4.4 (28/02/2025)
%
% Author: 
% Guillermo Jimenez (memo.notess1@gmail.com)
% 
% License:
% Creative Commons CC BY 4.0
% --------------------------------------------------------
%%%%%%%%%%%%%%%%%%%%%%%%%%%%%%%%%%%%%%%%%%%%%%%%%%%%%%%%%%%

\documentclass[9pt,a4paper,column,twoside]{tau-class/tau}
\usepackage[english]{babel}

%% Spanish babel recomendation
% \usepackage[spanish,es-nodecimaldot,es-noindentfirst]{babel} 

%% Draft watermark
% \usepackage{draftwatermark}

%----------------------------------------------------------
% TITLE
%----------------------------------------------------------

\journalname{Intro Microelectronic Circuits with Laboratory}
\title{Lab Name}

%----------------------------------------------------------
% AUTHORS, AFFILIATIONS AND PROFESSOR
%----------------------------------------------------------

\author[1]{Author One}
\author[1]{Author Two}
\author[1]{Author Three}

%----------------------------------------------------------

\affil[1]{Franklin W. Olin College of Engineering}

%----------------------------------------------------------
% FOOTER INFORMATION
%----------------------------------------------------------

\institution{Franklin W. Olin College of Engineering}
\footinfo{Intro Microelectronic Circuits with Laboratory}
\theday{02-09-2026}
\leadauthor{TBD}

%----------------------------------------------------------

\begin{document}
		
    \maketitle 
    \thispagestyle{firststyle} 
    % \tauabstract 
    % \tableofcontents
    % \linenumbers 
    
%----------------------------------------------------------

\section{Experiment 1}
	Color Code Resistance Value: 1.5 k$\Omega$ with a 5\% tolerance. Keithley Measured Resistance Value: 1.48377 k$\Omega$

\section{Experiment 2}
	Within a single chip, the standard deviation of the resistance between the isolated resistors was about 0.56\% with a range of 0.73. This is within the $\pm$2\% tolerance of the chip.

    \begin{table}
        \centering
        \begin{tabular}{ccc}
            1 & 9.9724 k$\Omega$\\
            2 & 9.9661 k$\Omega$\\
            3 & 9.9465 k$\Omega$\\
            4 & 9.9510 k$\Omega$\\
            5 & 9.9445 k$\Omega$\\
            6 & 9.9475 k$\Omega$\\
            7 & 9.9566 k$\Omega$\\
            8 & 9.9571 k$\Omega$\\
            9 & 9.9556 k$\Omega$\\
            10 & 9.9499 k$\Omega$\\
            11 & 9.9645 k$\Omega$\\
            12 & 9.9632 k$\Omega$\\
            13 & 9.9512 k$\Omega$\\
            14 & 9.9515 k$\Omega$\\
            15 & 9.9562 k$\Omega$\\
            16 & 9.9518\\
        \end{tabular}
        \caption{Caption}
        \label{tab:placeholder}
    \end{table}

%----------------------------------------------------------

\end{document}